\documentclass[letterpaper]{article}
\usepackage{aaai23}

\usepackage{times}
\usepackage{helvet}
\usepackage{courier}
\usepackage[hyphens]{url}
\usepackage{graphicx}
\usepackage{amsmath}
\usepackage{amssymb}
\usepackage{url}
\setlength{\emergencystretch}{2em}

\frenchspacing
\setlength{\pdfpagewidth}{8.5in}
\setlength{\pdfpageheight}{11in}

\title{Deep Learning for Network Intrusion Detection on NSL-KDD}

\author{
Aykhan Mammadli, \
Electrical and Computer Engineering, New York University \
\url{am15226@nyu.edu}
}

\begin{document}
\maketitle

\begin{abstract}
Intrusion Detection Systems classify network connections as benign or malicious.
We study binary intrusion detection on the NSL-KDD benchmark and compare a classical baseline like logistic regression with a multilayer perceptron as known as MLP. Because missed attacks are costly and
IDS datasets are often class-imbalanced, we emphasize attack-class recall and F1-score and test
imbalance-aware training (balanced class weights / weighted loss). On the NSL-KDD test set, the best
model is an MLP with \texttt{pos\_weight} achieving Recall=0.688 and F1=0.793, narrowly edging the unweighted MLP
(Recall=0.680, F1=0.791) and outperforming logistic regression
(Recall=0.591, F1=0.718). Source code: (\url{https://github.com/AykhanJ/DeepLearningProject/tree/main/IDS_Project})
\end{abstract}

\section{1. Introduction}
Networked systems face attacks such as probing, denial-of-service, and credential abuse. An Intrusion
Detection System aims to identify malicious activity from observed traffic while keeping false alarms
manageable. The most costly errors are considered false negatives in many security settings. This
motivates evaluating models with attack-class recall and F1-score rather than accuracy alone. Another practical
challenge is data imbalance. Data imbalance means, benign traffic may dominate, and certain attack patterns can be under-represented.
As a result, IDS evaluation typically emphasizes recall/F1 for the attack class and tests robustness techniques
that mitigate imbalance.

Our project formulates IDS as a supervised binary classification task using the NSL-KDD dataset, a widely used
benchmark derived from KDD Cup 99 \cite{tavallaee2009kdd}. 

Additionally, Our repository includes a README with step-by-step commands to reproduce preprocessing, training, and table/figure generation. 

We compare a strong classical baseline such as logistic regression against a simple deep learning model, MLP. To strengthen the methodology, we
add an imbalance-handling extension and empirically evaluate its effect on both model families. Our main finding is
that the MLP improves test-set recall and F1 over logistic regression. But imbalance weighting provides only small
gains for logistic regression and only marginal gains for the MLP in this setting.

\section{2. Background and Related Work}
\vspace{10mm}
\subsection{2.1 Benchmark Datasets for IDS}
KDD Cup 99 and its variant NSL-KDD have been widely used in IDS research. However, the original KDD Cup 99 dataset
contains redundant records that can bias training and inflate evaluation scores. NSL-KDD was proposed to mitigate
these issues by reducing redundancy and providing a more meaningful benchmark for comparing methods \cite{tavallaee2009kdd}.

NSL-KDD still remains a common reference point for controlled evaluations, despite it is not a modern enterprise traffic dataset. 
\vspace{5mm}
\subsection{2.2 Classical Machine Learning Approaches}
Traditional IDS approaches often rely on classical machine learning models trained on engineered connection features,
including decision trees, support vector machines, and logistic regression. These models are attractive because they
are relatively simple, fast to train, and can be easier to interpret than deep networks. Logistic regression in
particular is a strong baseline for linearly separable problems and is commonly used to establish a minimum-performance
reference.
\vspace{5mm}
\subsection{2.3 Deep Learning for IDS}
Deep learning approaches aim to capture non-linear feature interactions and reduce reliance on hand-crafted decision rules.
Shone et al.\ report competitive results with deep-learning-based intrusion detection and highlight benefits of learned
representations for detection \cite{shone2018deepids}. 

Other work studies deep and recurrent architectures for IDS, including
RNN-based intrusion detection \cite{yin2017deep}, and deep models applied to network traffic features \cite{javaid2016deep}.
\subsection{2.4 Architectural Choice}
We use an MLP because NSL-KDD provides fixed-length, connection-level tabular features after one-hot encoding and standardization. An MLP is a natural baseline for learning non-linear interactions among these heterogeneous attributes. In contrast, models designed for sequences are most appropriate when raw packet streams or time-ordered flow records are available, which this benchmark does not directly provide. A strong non-neural alternative for tabular IDS is tree-based ensembles (e.g., Random Forest or gradient boosting), which often perform competitively on structured features. We leave this comparison for future work.


Autoencoder ensembles have also been proposed for online anomaly detection. It emphasizes adaptive representations and
streaming settings \cite{mirsky2018kitsune}.
\vspace{5mm}
\subsection{2.5 Imbalance and Evaluation Metrics}
A persistent issue in IDS datasets is class imbalance. Imbalance-aware learning (class weighting, resampling, or loss
reweighting) is often used to improve detection of rare malicious events. Because missed attacks can be costly, IDS papers
frequently report recall and F1 for the attack class in addition to accuracy, and confusion matrices are important to
understand false negatives and false positives.
\vspace{5mm}
\section{3. Dataset and Preprocessing}
We use NSL-KDD, where each instance corresponds to a network connection described by mixed feature types, including
categorical attributes (for example, protocol type, service, flag) and numeric attributes (byte counts, error rates, traffic statistics and etc.). We convert original labels into a binary target: normal (0) versus attack (1). The official
dataset provides separate train and test files; we further split the training file to create a validation set.
\vspace{3mm}
\subsection{Preprocessing Pipeline}
We apply the following preprocessing steps:
\begin{itemize}
\item \textbf{Categorical encoding:} one-hot encode categorical attributes. Unknown categories are ignored at inference.
\item \textbf{Feature scaling:} standardize numeric features using training statistics only (mean 0, variance 1).
\item \textbf{Data splits:} create a stratified validation split from the training file (15\% validation). The official test
file is used only for final evaluation to prevent tuning on test data.
\end{itemize}
After preprocessing, each example becomes a fixed-dimensional numeric vector used by both logistic regression and the MLP.
We also save the fitted preprocessor to ensure reproducibility.

\vspace{10mm}
\section{4. Methodology}
\vspace{3mm}
\subsection{Baseline: Logistic Regression}
We train logistic regression on the processed feature vectors using the SAGA solver with a maximum of 500 iterations.
We evaluate two variants: (a) unweighted training and (b) balanced class weights via \texttt{class\_weight=balanced},
which reweights samples inversely proportional to class frequency.

\subsection{4.1 Deep Model: Multilayer Perceptron}
Our deep model is a feedforward MLP with hidden layer sizes [256, 128, 64], ReLU activations, and dropout 0.2. The
network outputs a single logit and is trained with binary cross-entropy with logits. We use Adam with learning rate
$10^{-3}$ and batch size 512. Early stopping is applied based on validation F1-score with patience 5; the best checkpoint
by validation F1 is used for test evaluation.

\subsection{4.2 Imbalance Handling Extension}
To address class imbalance as a methodological extension, we evaluate (i) balanced class weights for logistic regression and
(ii) a weighted loss for the MLP using a positive-class weight \texttt{pos\_weight} computed from the training split as the
ratio of negative to positive examples. This increases the penalty for misclassifying attacks.

\subsection{4.3 Hyperparameter Selection and Training Details}
Hyperparameters were selected using the validation set with attack-class F1 as the primary model-selection metric. For logistic
regression we fixed common stable settings (SAGA solver, max\_iter=500). For the MLP we chose a small architecture suitable for
tabular data and tuned only lightweight parameters (dropout and early stopping). We use a fixed random seed for reproducibility.
All models share the same preprocessing, splits, and evaluation procedure to ensure a fair comparison.

\subsection{4.4 Evaluation Metrics}
We report accuracy, precision, recall, and F1-score for the positive class (attack). Confusion matrices are also included to
interpret types of errors. Predictions use a 0.5 threshold on sigmoid probability (equivalently, logit $\ge 0$).
\vspace{10mm}
\section{5. Experiments and Results}
\vspace{3mm}
\subsection{5.1 Main Results}
Table~\ref{tab:results} reports test-set performance. Logistic regression achieves F1=0.718 and Recall=0.591. Applying balanced
class weights yields a modest improvement (F1=0.724, Recall=0.599). The unweighted MLP substantially outperforms both logistic
regression variants (F1=0.791, Recall=0.680).

Adding \texttt{pos\_weight} to the MLP yields a very small improvement in Recall and F1 (Recall=0.688, F1=0.793), while slightly reducing
precision (0.936 vs.\ 0.944). This suggests the weighted loss mildly shifts predictions toward the positive class, improving missed-attack rate
at the cost of a small increase in false positives.

% Auto-generated by scripts/make_results_table.py
\begin{table}[t]
\centering
\begin{tabular}{lcccc}
\hline
Model & Acc & Prec & Recall & F1 \\
\hline
LogReg & 0.736 & 0.915 & 0.591 & 0.718 \\
LogReg + class\_weight & 0.740 & 0.915 & 0.599 & 0.724 \\
MLP & 0.795 & 0.944 & 0.680 & 0.791 \\
MLP + pos\_weight & 0.795 & 0.936 & 0.688 & 0.793 \\
\hline
\end{tabular}
\caption{Performance on NSL-KDD (attack is positive class).}
\label{tab:results}
\end{table}


\begin{figure}[t]
\centering
\includegraphics[width=\columnwidth]{mlp_training_curve.pdf}
\caption{MLP training loss and validation F1 across epochs (early stopping selects the best validation F1).}
\label{fig:curve}
\end{figure}


\subsection{5.2 Discussion: Validation vs.\ Test Gap}
Validation performance is substantially higher than test performance. The behavior is consistent with known differences
between NSL-KDD training and test distributions as the test set includes harder and less frequent patterns. Therefore, we emphasize
the official test-set results for final comparison and keep model selection restricted to the validation split.

\section{6. Conclusion}
We originally intended to determine if a simple deep learning model could outperform a classical baseline on imbalanced network traffic data.
We implemented an end-to-end IDS classification pipeline on NSL-KDD and compared logistic regression against a simple deep MLP.
The strongest test performance was achieved by the MLP with \texttt{pos\_weight} (Recall=0.688, F1=0.793), improving over the baseline
(LogReg: Recall=0.591, F1=0.718). We also evaluated imbalance-aware training as a methodological extension: it provided small gains for logistic regression
and only marginal gains for the MLP under the NSL-KDD test distribution. Future work could explore threshold tuning to trade off recall vs.\ false alarms,
alternative imbalance methods, and cross-dataset evaluation to better estimate real-world generalization.

\bibliographystyle{aaai23}
\bibliography{references}

\end{document}
